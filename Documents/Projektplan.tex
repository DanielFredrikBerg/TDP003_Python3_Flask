\documentclass{TDP003mall}
\usepackage[utf8]{inputenc}
\usepackage[swedish]{babel}
\usepackage[xcolor]{}
\usepackage{enumitem}

\newcommand{\version}{Version 1.2}
\author{Daniel Huber, \url{danhu849@liu.se}\\
  Jens Öhnell, \url{jenoh242@liu.se}}
\title{Projektplan}
\date{2020-09-23}
\rhead{Daniel Huber\\
Jens Öhnell}

\renewcommand*\contentsname{Innehållsförteckning}
\setlist{topsep=0pt, leftmargin=*}

\begin{document}
\projectpage

\tableofcontents
\newpage



\section{Revisionshistorik}
\begin{table}[!h]
\begin{tabularx}{\linewidth}{|l|X|l|}
\hline
Ver. & Revisionsbeskrivning & Datum \\\hline
1.2 & Innehållsförteckning och tekniker & 230920\\\hline
1.1 & Aktiviteter lagts till för varje deadline & 220920\\\hline
1.0 & Projektplan 1:a utkast & 200922 \\\hline
\end{tabularx}
\end{table}


\section{Introduktion}
Projektets mål utgörs av att skapa, presentera och underhålla en webbaserad portfolio. Där presenteras de projekt som vi i och utanför universitetet ska färdigställa under de kommande 3 åren. Den kompletta kravspecifikationen skrevs av programledningen och återfinns i dokumentet \textit{Systemspecifikation av portfoliosystemet}. En sammanfattning finns i detta dokument.

För de avsnitt där det antingen är vagt eller inte alls specificerat vad eller hur något ska göras förväntas det att studenten tar egna initiativ. Exempel på detta är utseendet på användargränssnittet där nästintill total frihet ges.

\section{Tekniker}
För att uppfylla kravspecifikationen för detta projekt kommer följande tekniker användas:
\begin{itemize}
\item Python3
\item Git
\item HTML5
\item CSS3
\item Flask
\item Jinja2
\item venv
\item JSON
\item Latex
\end{itemize}

Pythonpaketet venv för att möjlggöra utveckling i virtuell miljö samt minimera risken av paketkonflikter.
Pythonpaketet Flask tillhandahåller debugger, möjliggör bindandet av python3 funktioner till URL paths (API) och agerar webserver.
Pythonpaketet Jinja2 för HMTL5 och CSS3 templates.
Versionshantering sköts med git.
En JSON fil, data.json representerar datalagret.
Dokumentationen skrivs i Latex.

\section{Kravspecifikation}
Projektet utgörs av två delar. Dels det slutanvändaren kan se, det presentativa, och dels det som denne inte kan se, datalagringen och datahämtningen.

Av webplatsen krävs det att den utrustas med fyra html-sidor skrivna i HTML5 och CSS3. En huvudsida/första sida som antingen kan vara statiskt eller dynamisk samt 3 stycken dynamiska sidor. De tre sistnämnda utgörs av en söksida, en projektsida och en tekniksida. Krav finns att det på huvudsidan visas bilder. På söksidan kan projekt sorteras efter projektens id samt efter bokstavsordning och ålder med hjälp av knappar. På respektive projektsida visas fullständig information om det specifika projektet tillsammans med en större passande bilder. Om projektet inte finns visas relevant felkod i.e. status 404 'This page does not exist'. På tekniksidan visas information om projekten utifrån vilka tekniker som använts. Ett ambitionsmål är att kunna visa i hur stor omfattning teknikerna använts i projekten. Varje projekt ska i listningar på söksidan och tekniksidan visas med en liten bild bredvid sig. Bildtext måste finnas till varje bild. Vid fel ska dessa hanteras på ett, för slutanvändaren, informativt sätt så att denne kan förstå vad som gått fel.

Datalagret utgörs av JSON-kod med UTF-8 teckenkodning i JSON-filen data.json. Varje projektinstans i JSON-koden utgörs av projektnamn, projekt-id i form av ett unikt heltalsnummer, startdatum, slutdatum, kurskod, kursnamn, kurspoäng, nyttjade tekniker, sammanfattning, full beskrivning, liten och stor bild, antal gruppmedlemmar och länk till projektsida. JSON-koden manipuleras med hjälp av ett API utgörande av sex stycken standardiserade funktioner. Samtliga namn skrivs på engelska. Funktioner som ska implementeras: load(filename), get\_project\_count(db), get\_project(db, id), search(db, sort\_by='start date', sort\_order='desc', techniques=None, search=None), get\_techniques(db), get\_technique\_stats(db). För fullständig specifikation hänvisas läsaren till dokumentet \textit{Application Programming Interface (API)} som finns på kurshemsidan.

Vid sökning av godtyckliga termer om projektets information genereras träffar och slutanvändaren presenteras med en lista där det mest förmodade objektet presenteras högst upp eller längst ner. Tekniksidan kan sorteras på använda tekniker. Det ska noteras att funktionaliteten ska möjliggöra sökning på ett ord, sortering och filtrering av tekniker. Sökningarna ska ske samtidigt. Datum är i formatet ISO 8601. Förändringar i data.json filen ska för användaren presenteras direkt utan nödvändig omstart av webbservern. En frivillighet och alltså inte ett krav är att lägga till en administrativ sida för redigering av data.

Projektet versionshanteras med git.

Vid systemets slutförande testas systemets funktioner av två personer som ej ingått i utvecklarteamet. Systemtesten och uppkomna fel dokumenteras och ska vara åtgärdade i den slutgiltiga versionen av projektet. Testerna skrivs sedan in i systemdokumentationen.

\section{Riskbedömning och åtgärder}
Den största risken mot att inte bli klar i tid är sjukdom som redan drabbat en del av teamet en gång när detta skrivs (23/9). Det löstes av att den andra teammedlemmen fick rusha arbetet innan deadline. Till vår hjälp har vi fått betyget med beröm för tidigare inlämningar. Detta gör att vi i värsta fall har möjligheten att lämna in senare än deadline så länge detta kommuniceras till handledare innan. Ambitionen är dock att aldrig behöva nyttja detta.

Vid skador, sjukdomar eller andra traumatiska händelser av allvarlig sort har familjemedlemmar blivit instruerade att informera handledare så fort som möjligt. Detta för att resurser från universitetet i den mån det är möjligt kan sättas in tidigt.

Beroende på tidsåtgången att inhämta ny kunskap kan nya flaskhalsar uppstå längre bort i tiden av saker vi inte kunnat förutse då vi saknar kunskap. För att underlätta har vi satt ambitionsmål att läsa om och testa tekniker en eller två veckor innan de introduceras första gången. Dessa har dock lägre prioritering än det som måste göras innan nästa deadline, men finns som en påminnelse att om tid finns så kan den användas till att underlätta framtida deadlines.

Slutligen finns risken att bli upptagen eller fastna med andra uppgifter i andra kurser. Denna projektplan täcker bara saker som måste göras för detta projekt. Risk finns att för stort fokus läggs på denna kurs så att andra kurser blir försenade och stör längre in i tiden. Vi har satt in åtgärder mot detta genom att minst en gång i veckan ha möte och diskutera ens egna deadlines för veckan i alla kurser. Det gör det lättare att se vem som kan göra vad när.

\section{Tidsplanering vecka för vecka}
Projektplanen uppdelas i veckor för lättare översikt. För varje vecka presenteras en mer specificerad uppskattning av tidsåtgång för aktiviteter som behöver vara klara innan deadline. Tidsåtgången för deadlinen inkluderar också tiden för dess aktiviteter. För de aktiviteter som avklarats visas också den verkliga tidsåtgången samt datum för färdigställande. Varje veckas deadlines presenteras i tabeller. Ifall det saknas deadlines för någon vecka så saknas också tabell.

\subsection{Vecka 37}
\begin{tabular}{|l|l|l|l|l|}
  \hline
  Deadline Datum & Uppskattad tid & Tidsåtgång & Beskrivning & Kännedom\\ [0.5ex]
  \hline
  Torsdag 10/9 & 6h & 7h & Planeringsdokument & God\\
  \hline
\end{tabular}

Tidsåtgången för planeringsdokumentet utgjordes av 6 timmar. Av dessa ägnades:
\begin{itemize}
 \item 8/9 3 tim sammanställande av vad ett planeringsdokument ska innehålla.
 \item 8/9 2 tim skrivande av dokumentet.
 \item 9/9 30 min åt att figurera ut hur latex dokumentet skulle kompileras.
 \item 9/9 1 tim att ta reda på hur en git \texttt{-{}-}hard-reset reverseras.
 \item 9/9 30 min att figurera ut hur tabeller skrivs i latex.
\end{itemize}

\subsection{Vecka 38}
\begin{tabular}{|l|l|l|l|l|}
  \hline
  Deadline Datum & Uppskattad tid & Tidsåtgång & Beskrivning & Kännedom\\ [0.5ex]
  \hline
  Torsdag 17/9 & 4h & 9 tim 15 min & Grundläggande Installationsmanual & God\\
  \hline
  Torsdag 17/9 & 6h & 7 tim 29 min & Lofi-prototyp & God\\
  \hline
\end{tabular}

Den Grundläggande installationsmanualen färdigställdes på 9 tim och 5 min:
\begin{itemize}
\item 14/9 1 tim 40 min Dokumentet skrevs
\item 14/9 20 min Bash-script som tog bort onödiga latex filer skrevs.
\item 14/9 2 tim lära sig kompilera .tex dokument innehållandes bilder samt lägga in dem och läsa på om paket.
\item 15/9 4 tim Dokumentet skrevs
\item 15/9 1 tim att ta reda på alla kommandon som visar att installtion gått som förväntat.
\item 15/9 15 min Dokumentet rättades och bildernas plats justerades.\\
\end{itemize}

Lofi-prototypen färdigställdes på 7 tim och 29 min:
\begin{itemize}
\item 16/9 2 tim Utkast snabbskissades för samtliga sidor.
\item 16/9 3 tim Skissarna förbättrades och renskrevs för hand.
\item 17/9 2 tim 9 min De renskrivna skisserna scannades in och sammanställdes med förklaringar i ettlatex dokument.
\item 17/9 20 min Dokumentet rättades och bildernas plats justerades.
\end{itemize}


\subsection{Vecka 39}
\begin{tabular}{|l|l|l|l|l|}
  \hline
  Deadline Datum & Uppskattad tid & Tidsåtgång & Beskrivning & Kännedom\\ [0.5ex]
  \hline
  Torsdag 24/9 & 24h & 8h hittills & 1:a V Gemensam installationsmanual & God\\
  \hline
  Torsdag 24/9 & 24h & ca 4h hittills & Projektplan, Första utkast & God\\
  \hline
\end{tabular}

1:a Versionen Gemensam installationsmanual färdigställdes på 8h hittills:
\begin{itemize}
\item 21/9 20 min Skapa branch från master för development.
\item 21/9 5 tim 31 min Skriva READMEs i markdown i både hemkatalagen och manuals katalogen. Fixa lättförståelig katalogstruktur. Felsöka initiella svårigheter att push:a upp filer.
\item 22/9 3 tim skriva issues, hjälp med felsökning av andras problem och evaluera färdiga issues.\\
\end{itemize}

  Projektplan 1:a utkast färdigställdes på 4h hittills:
  \begin{itemize}
\item 17/9 1 tim Definera 2-3 aktiviteter inför varje deadline.
\item 22/9 4 tim Skriva in aktiviteterna i dokumentet samt bestämma upplägg.
  \item 23/9 2 tim Anteckningar renskrevs och sorterades in i dokumentet.
\item Sätta ambitionsmål ifall projektet blir klart i förtid.
\item Komma fram till riskbedömning och åtgärder.
\end{itemize}

\subsection{Vecka 40}
\begin{tabular}{|l|l|l|l|l|}
  \hline
  Deadline Datum & Uppskattad tid & Tidsåtgång & Beskrivning & Kännedom\\ [0.5ex]
  \hline
  Torsdag 1/10 & 1h &  & Bidra med icke-trivial förbättring git eller tester & God o inget\\
  \hline
  Torsdag 1/10 & 1-2h &  & Korrigera eventuella brister installationsmanualen & okänt\\
  \hline
  Torsdag 1/10 & 1-2h &  & Korrigera Brister, Projektplanen & Beror på\\
  \hline
  Torsdag 1/10 & 24h &  & Datalagret Godkänt & Vag\\
  \hline
\end{tabular}

Bidra med icke-trivial förbättring - installationsmanual eller tester\\
Korrigera eventuella brister gemensamma installationsmanualen\\
Korrigera Brister, Projektplanen\\
Datalagret Godkänt\\

\subsection{Vecka 41}


\subsection{Vecka 42}
\begin{tabular}{|l|l|l|l|l|}
  \hline
  Deadline Datum & Uppskattad tid & Tidsåtgång & Beskrivning & Kännedom\\ [0.5ex]
  \hline
  Torsdag 15/10 & - &  & Portfolion Publicerad & Vag\\
  \hline
  Torsdag 15/10 & 3h &  & Systemdemonstration & Vag\\
  \hline
  Torsdag 15/10 & 8h &  & 1:a Versionen Systemdokumentation & Vag\\
  \hline
\end{tabular}

Portfolion Publicerad\\
1:a Versionen Systemdokumentation\\

\subsection{Vecka 43}
\begin{tabular}{|l|l|l|l|l|}
  \hline
  Deadline Datum & Uppskattad tid & Tidsåtgång & Beskrivning & Kännedom\\ [0.5ex]
  \hline
  Torsdag 22/10 & 8h &  & Testdokumentation inlämnad & Vag\\
  \hline
  Torsdag 22/10 & 6h &  & Individuellt Reflektionsblad & Vag\\
  \hline
  Torsdag 22/10 & 2-3h &  & Korrigerat event. brister i systemdokumentationen & Vag\\
  \hline
\end{tabular}

Testdokumentation inlämnad\\
Individuellt Reflektionsblad\\
Korrigerat eventuella brister i systemdokumentationen\\


\subsection{Schema/Deadlines}

\begin{tabular}{|l|l|l|l|l|}
  \hline
  Vecka & Deadline Datum & Uppskattad tid & Beskrivning & Kännedom\\ [0.5ex]
  \hline
  & Lördag 12/9 & 3h & Sätta emacs i python-mode & Vag\\
  \hline
  & Söndag 13/9 & 2h & Inkorporera Magit i Emacs & Vag\\
  \hline
  38 & Tisdag 15/9 & 3h & Bekantskap med Seleniumhq.org & Vag\\
  \hline
  & Torsdag 17/9 & 6h & Lofi-prototyp & God\\
  \hline
  & Torsdag 17/9 & 4h & Grundläggande Installationsmanual & God\\
  \hline
  39 & Torsdag 24/9 & 24h & Projektplan, Första utkast & Vag\\
  \hline
  & Torsdag 24/9 & 24h & 1:a Versionen gemensam installationsmanual & God\\
  \hline
  & Fredag 25/9 & 2h & Flask o Jinja2 Föreläsning & Vag\\
  \hline
  & Söndag 27/9 & 6h & Ha deploy:at första testhemsidan med flask & Vag\\
  \hline
  & Söndag 27/9 & 12h & Fatta Jinja2 & Vag\\
  \hline
  40 & Torsdag 1/10 & 1h & Bidra med icke-trivial förbättring git eller tester & God o inget\\
  \hline
  & Torsdag 1/10 & 1-2h & Korrigera eventuella brister installationsmanualen & okänt\\
  \hline
  & Torsdag 1/10 & 1-2h & Korrigera Brister, Projektplanen & Beror på\\
  \hline
  & Torsdag 1/10 & 24h & Datalagret Godkänt & Vag\\
  \hline
  42 & Torsdag 15/10 & - & Portfolion Publicerad & Vag\\
  \hline
  & Torsdag 15/10 & 3h & Systemdemonstration & Vag\\
  \hline
  & Torsdag 15/10 & 8h & 1:a Versionen Systemdokumentation & Vag\\
  \hline
  43 & Torsdag 22/10 & 8h & Testdokumentation inlämnad & Vag\\
  \hline
  & Torsdag 22/10 & 6h & Individuellt Reflektionsblad & Vag\\
  \hline
  & Torsdag 22/10 & 2-3h & Korrigerat event. brister i systemdokumentationen & Vag\\
  \hline
  \hline
  Alla & & 20min/dag & Dokumentera Dagbok & God\\
  \hline
\end{tabular}

\subsection{Tidsåtgång - Presentativ Del}
\begin{tabular}{|c|l|}
  \hline
  5h & Statisk eller dynamisk sida med bilder på /\\
  \hline
  8h & Dynamisk sida som sorterar och listar projekten på /list\\
  \hline
  2h & Visar fullständig infosida för specifikt proj med id på /project/id\\
  \hline
  3h & Sammanställning av alla project baserat på använda tekniker på /techniques\\
  \hline
\end{tabular}

\subsection{Tidsåtgång - Funktioner i datalagret}
Nedan listas den specifika tidsåtgången för implementeringar av respektive funktion i datalagret.

\begin{tabular}{|c|l|}
  \hline
  Tid & Funktion\\
  \hline
  2h & load\\
  \hline
  1h & get\_project\_count\\
  \hline
  1h & get\_project\\
  \hline
  3h & search\\
  \hline
  2h & get\_techniques\\
  \hline
  2h & get\_technique\_stats\\
  \hline
\end{tabular}

\section{Riskbedömning och åtgärder}


\end{document}

\documentclass{TDP003mall}
\usepackage[utf8]{inputenc}
\usepackage[swedish]{babel}


\newcommand{\version}{Version 1.0}
\author{Daniel Huber, \url{danhu849@liu.se}\\
  Jens Öhnell, \url{jenoh242@liu.se}}
\title{Planeringsdokument}
\date{2020-09-15}
\rhead{Daniel Huber\\
Jens Öhnell}



\begin{document}
\projectpage
\section{Revisionshistorik}
\begin{table}[!h]
\begin{tabularx}{\linewidth}{|l|X|l|}
\hline
Ver. & Revisionsbeskrivning & Datum \\\hline
1.0 & Projektplan 1:a utkast & 200915 \\\hline
\end{tabularx}
\end{table}


\section{Introduktion}
Projektets mål utgörs av att skapa, presentera och underhålla en webbaserad portfolio. Där presenteras de projekt som vi i och utanför universitetet ska färdigställa under de kommande 3 åren. Kravspecifikationen skrevs av programledningen och återfinns i dokumentet \textbf{DOkumentets namn här!}. För de avsnitt där det antingen är vagt eller inte alls specificerat vad eller hur något ska göras förväntas det att studenten tar egna initiativ. Exempel på detta är utseendet på användargränssnittet där nästintill total frihet ges.


\section{Kravspecifikation}
Projektet utgörs av två delar. Dels det slutanvändaren kan se, det presentativa, och dels det som denne inte kan se, datalagringen och datahämtningen.

Av webplatsen krävs det att den utrustas med fyra html-sidor skrivna i HTML5 och CSS3. En huvudsida/första sida som antingen kan vara statiskt eller dynamisk samt 3 stycken dynamiska sidor. De tre sistnämnda utgörs av en söksida, en projektsida och en tekniksida. Krav finns att det på huvudsidan visas biler. På söksidan finns funktionaliteten att kunna sortera projekt på projektets id samt möjliggöra sökning med hjälp av söktermer i ett formulär. På respektive projektsida visas fullständig information om det specifika projektet tillsammans med en stor passande bild. Om projektet inte finns visas relevant felkod i.e. status 404 'This page does not exist'. På tekniksidan visas information om projekten utifrån vilka tekniker och i hur stor omfattning dessa använts i projekten. Varje projekt ska i listningar på söksidan och tekniksidan visas med en liten bild bredvid sig. Bildtext måste finnas till varje bild. Vid fel ska dessa hanteras på ett, för slutanvändaren, informativt sätt så att denne kan förstå vad som gått fel.

Datalagret utgörs av JSON-kod med UTF-8 teckenkodning i JSON-filen data.json. Varje projektinstans i JSON-koden utgörs av projektnamn, projekt-id i form av ett unikt heltalsnummer, startdatum, slutdatum, kurskod, kursnamn, kurspoäng, nyttjade tekniker, sammanfattning, full beskrivning, liten och stor bild, antal gruppmedlemmar och länk till projektsida. JSON-koden manipuleras med hjälp av ett API utgörande av sex stycken standardiserade funktioner. Samtliga namn skrivs på engelska. Funktioner som ska implementeras: load(filename), get\_project\_count(db), get\_project(db, id), search(db, sort\_by='start date', sort\_order='desc', techniques=None, search=None), get\_techniques(db), get\_technique\_stats(db). För komplett specifikation hänvisas läsaren till \textbf{Systemkravspecifikationens namn}.

Vid sökning med godtyckliga termer om projektets information genereras träffar och slutanvändaren presenteras med en lista där det mest förmodade objektet presenteras högst upp eller längst ner. Söklistan kan sorteras på använda tekniker. Det ska noteras att funktionaliteten ska möjliggöra sökning på ett ord, sortering och filtrering av tekniker. Sökningarna ska ske samtidigt. Datum är i formatet ISO 8601. Förändringar i data.json filen ska för användaren presenteras direkt utan nödvändig omstart av webbservern. En frivillighet och alltså inte ett krav är att lägga till en administrativ sida för redigering av data.

Projektet versionshanteras med git.

Vid systemets slutförande testas systemets funktioner av två personer som ej ingått i utvecklarteamet. Systemtesten och uppkomna fel dokumenteras och ska vara åtgärdade i den slutgiltiga versionen av projektet. Testerna skrivs sedan in i systemdokumentationen.


\section{Tidsplanering vecka för vecka}

Nedan följer en grov uppskattning av tidsåtgång för de olika momenten nödvändiga att behandla innan projektet kan anses slutfört. Tidsuppskattningen baseras på att gruppens medlemmar dels skummat igenom innehållet för respektive ämne och sedan efter diskussion kommit fram till en förväntad tidsåtgång för att bemästra sagt ämne. Uppskattningen anses vara grov då det utgås ifrån nuvarande kunskap idag (2020-09-09).

\subsection{Schema/Deadlines}

\begin{tabular}{|l|l|l|l|l|}
  \hline
  Vecka & Deadline Datum & Uppskattad & Beskrivning & Kunskap \\ [0.5ex]
  \hline
  37 & Torsdag 10/9 & 6h & Planeringsdokument & God\\
  \hline
  & Lördag 12/9 & 3h & Sätta emacs i python-mode & Vag\\
  \hline
  & Söndag 13/9 & 2h & Inkorporera Magit i Emacs & Vag\\
  \hline
  38 & Tisdag 15/9 & 3h & Bekantskap med Seleniumhq.org & Vag\\
  \hline
  & Torsdag 17/9 & 6h & Lofi-prototyp & God\\
  \hline
  & Torsdag 17/9 & 4h & Grundläggande Installations manual & God\\
  \hline
  39 & Torsdag 24/9 & 24h & Projektplan, Första utkast & Vag\\
  \hline
  & Torsdag 24/9 & 24h & 1:a Versionen gemensam installationsmanual & God\\
  \hline
  & Fredag 25/9 & 2h & Flask o Jinja2 Föreläsning & Vag\\
  \hline
  & Söndag 27/9 & 6h & Ha deploy:at första testhemsidan med flask & Vag\\
  \hline
  & Söndag 27/9 & 12h & Fatta Jinja2 & Vag\\
  \hline
  40 & Torsdag 1/10 & 1h & Bidra med icke-trivial förbättring git eller tester & God o inget\\
  \hline
  & Torsdag 1/10 & 1-2h & Korrigera eventuella brister installationsmanualen & okänt\\
  \hline
  & Torsdag 1/10 & 1-2h & Korrigera Brister, Projektplanen & Beror på\\
  \hline
  & Torsdag 1/10 & 24h & Datalagret Godkänt & Vag\\
  \hline
  42 & Torsdag 15/10 & - & Portfolion Publicerad & Vag\\
  \hline
  & Torsdag 15/10 & 3h & Systemdemonstration & Vag\\
  \hline
  & Torsdag 15/10 & 8h & 1:a Versionen Systemdokumentation & Vag\\
  \hline
  43 & Torsdag 22/10 & 8h & Testdokumentation inlämnad & Vag\\
  \hline
  & Torsdag 22/10 & 6h & Individuellt Reflektionsblad & Vag\\
  \hline
  & Torsdag 22/10 & 2-3h & Korrigerat event. brister i systemdokumentationen & Vag\\
  \hline
  \hline
  Alla & & 20min/dag & Dokumentera Dagbok & God\\
  \hline
\end{tabular}

\subsection{Tidsåtgång - Presentativ Del}
\begin{tabular}{|c|l|}
  \hline
  5h & Statisk eller dynamisk sida med bilder på /\\
  \hline
  8h & Dynamisk sida som sorterar och listar projekten på /list\\
  \hline
  2h & Visar fullständig infosida för specifikt proj med id på /project/id\\
  \hline
  3h & Sammanställning av alla project baserat på använda tekniker på /techniques\\
  \hline
\end{tabular}

\subsection{Tidsåtgång - Funktioner i datalagret}
Nedan listas den specifika tidsåtgången för implementeringar av respektive funktion i datalagret.

\begin{tabular}{|c|l|}
  \hline
  Tid & Funktion\\
  \hline
  2h & load\\
  \hline
  1h & get\_project\_count\\
  \hline
  1h & get\_project\\
  \hline
  3h & search\\
  \hline
  2h & get\_techniques\\
  \hline
  2h & get\_technique\_stats\\
  \hline
\end{tabular}

\end{document}

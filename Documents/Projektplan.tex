\documentclass{TDP003mall}
\usepackage[utf8]{inputenc}
\usepackage[swedish]{babel}
\usepackage[xcolor]{}
\usepackage{tabularx}
\usepackage{enumitem}
\usepackage{makecell}

% FRÅGA I MEJL: Underligt i innehållsförteckningen att veckoplaneringssidorna är felaktiga. 

% FRÅGA I MEJL: Samt att dokumentets sista sida är 10 av 9.

% FRÅGA I MEJL OM DET ÄR OK ATT LÅTA SOM VI REDAN GJORT SAKER UTAN ATT FAKTISKT HA GJORT DET (PASSIV FORM). Undvik att skriva projektplanen i "vi"-form. Det ska vara ett passivt dokument. Även undvika ord som "rusha" då det inte riktigt passar i sådana dokument.

%Saknar lite information om rutiner. Hur kommer ni programmera, hur mycket används git, när kontrollerar ni om funktioner fungerar på rätt sätt?

%Milstolpar bör vara mindre arbetsmässiga än deadlines, och därmed inte samma med tidigare datum.

%Ni saknar figurtexter till tabeller och då även referenser till dem också. Tack vare er layout är det relativt tydligt vad som tillhör pch refererar till tabeller, men bättre att göra definitiva referenser.

\newcommand{\version}{Version 1.0}
\author{Daniel Huber, \url{danhu849@liu.se}\\
	Jens Öhrnell, \url{jenoh242@liu.se}}
\title{Projektplan}
\date{2020-09-28}
\rhead{Daniel Huber\\
Jens Öhrnell}

\renewcommand*\contentsname{Innehållsförteckning}
\setlist{topsep=0pt, leftmargin=*}

\begin{document}
\projectpage

\tableofcontents

\newpage
\section{Revisionshistorik}
\begin{table}[!h]
\begin{tabularx}{\linewidth}{|l|X|l|}
\hline
  Ver. & Revisionsbeskrivning & Datum \\\hline
  1.1 & Korrigerat Projektplanens kompletteringar & 280920 \\\hline
  1.0 & Projektplan 1:a Utkast & 240920 \\\hline
\end{tabularx}
\end{table}


\section{Introduktion}
Projektets mål utgörs av att skapa, presentera och underhålla en webbaserad portfolio. Där presenteras de projekt som vi i och utanför universitetet ska färdigställa under de kommande 3 åren. Den kompletta kravspecifikationen skrevs av programledningen och återfinns i dokumentet \textit{Systemspecifikation av portfoliosystemet}. En sammanfattning finns i detta dokument.

För de avsnitt där det antingen är vagt eller inte alls specificerat vad eller hur något ska göras förväntas det att studenten tar egna initiativ. Exempel på detta är utseendet på användargränssnittet där nästintill total frihet ges.

\section{Tekniker}
För att uppfylla kravspecifikationen för detta projekt kommer följande tekniker användas:
\begin{itemize}
\item Python3
\item Git
\item HTML5
\item CSS3
\item Flask
\item Jinja2
\item venv
\item JSON
\item Latex
\end{itemize}

Pythonpaketet venv används för att möjlggöra utveckling i virtuell miljö samt minimera risken av paketkonflikter.
Pythonpaketet Flask tillhandahåller debugger, möjliggör bindandet av python3 funktioner till URL paths (API) och agerar webbserver.
Pythonpaketet Jinja2 för HMTL5 och CSS3 templates.
Versionshantering sköts med git.
En JSON fil, data.json representerar datalagret.
Dokumentationen skrivs i Latex.

\newpage

\section{Arbetsmetodik}


\section{Kravspecifikation}
Projektet utgörs av två delar. Dels det slutanvändaren kan se, det presentativa,
och dels det som denne inte kan se, datalagringen och datahämtningen.

Av webplatsen krävs det att den utrustas med fyra html-sidor skrivna i HTML5 och
 CSS3. En huvudsida/första sida som antingen kan vara statiskt eller dynamisk samt
 3 stycken dynamiska sidor. De tre sistnämnda utgörs av en söksida, en projektsida
 och en tekniksida. Krav finns att det på huvudsidan visas bilder. På söksidan kan
 projekt sorteras efter projektens id samt efter bokstavsordning och ålder med hjälp av knappar.
 På respektive projektsida visas fullständig information om det specifika
 projektet tillsammans med en större passande bilder. Om projektet inte finns visas
 relevant felkod i.e. status 404 'This page does not exist'. På tekniksidan visas
 information om projekten utifrån vilka tekniker som använts. Ett ambitionsmål är
 att kunna visa i hur stor omfattning teknikerna använts i projekten. Varje projekt
 ska i listningar på söksidan och tekniksidan visas med en liten bild bredvid sig
. Bildtext måste finnas till varje bild. Vid fel ska dessa hanteras på ett, för 
slutanvändaren, informativt sätt så att denne kan förstå vad som gått fel.

Datalagret utgörs av JSON-kod med UTF-8 teckenkodning i JSON-filen data.json. Varje
 projektinstans i JSON-koden utgörs av projektnamn, projekt-id i form av ett unikt
 heltalsnummer, startdatum, slutdatum, kurskod, kursnamn, kurspoäng, nyttjade tekniker
, sammanfattning, full beskrivning, liten och stor bild, antal gruppmedlemmar och
 länk till projektsida. JSON-koden manipuleras med hjälp av ett API utgörande av
 sex stycken standardiserade funktioner. Samtliga namn skrivs på engelska. Funktioner
 som ska implementeras: load(filename), get\_project\_count(db), get\_project(db
, id), search(db, sort\_by='start date', sort\_order='desc', techniques=None, search
=None), get\_techniques(db), get\_technique\_stats(db). För fullständig specifikation
 hänvisas läsaren till dokumentet \textit{Application Programming Interface (API
)} som finns på kurshemsidan.

Vid sökning av godtyckliga termer om projektets information genereras träffar och
 slutanvändaren presenteras med en lista där det mest förmodade objektet presenteras
 högst upp eller längst ner. Tekniksidan kan sorteras på använda tekniker. Det ska
 noteras att funktionaliteten ska möjliggöra sökning på ett ord, sortering och filtrering
 av tekniker. Sökningarna ska ske samtidigt. Datum är i formatet ISO 8601. Förändringar
 i data.json filen ska för användaren presenteras direkt utan nödvändig omstart av
 webbservern. En frivillighet och alltså inte ett krav är att lägga till en administrativ
 sida för redigering av data.

Projektet versionshanteras med git.

Vid systemets slutförande testas systemets funktioner av två personer som ej ingått i utvecklarteamet.
Systemtesten och uppkomna fel dokumenteras och ska vara åtgärdade i den slutgiltiga versionen av projektet.
Testerna skrivs sedan in i systemdokumentationen.

\section{Riskbedömning och åtgärder}
Den största risken mot att inte bli klar i tid är sjukdom som redan drabbat en del av teamet en gång när detta skrivs (23/9). Det löstes av att den andra teammedlemmen fick rusha arbetet innan deadline. Till vår hjälp har vi fått betyget med beröm för tidigare inlämningar. Detta gör att vi i värsta fall har möjligheten att lämna in senare än deadline så länge detta kommuniceras till handledare innan. Ambitionen är dock att aldrig behöva nyttja detta.

Vid skador, sjukdomar eller andra traumatiska händelser av allvarlig sort har familjemedlemmar blivit instruerade att informera handledare så fort som möjligt. Detta för att resurser från universitetet i den mån det är möjligt kan sättas in tidigt.

Beroende på tidsåtgången att inhämta ny kunskap kan nya flaskhalsar uppstå längre bort i tiden av saker vi inte kunnat förutse då vi saknar kunskap. För att underlätta har vi satt ambitionsmål att läsa om och testa tekniker en eller två veckor innan de introduceras första gången. Dessa har dock lägre prioritering än det som måste göras innan nästa deadline, men finns som en påminnelse att om tid finns så kan den användas till att underlätta framtida deadlines.

Slutligen finns risken att bli upptagen eller fastna med andra uppgifter i andra kurser. Denna projektplan täcker bara saker som måste göras för detta projekt. Risk finns att för stort fokus läggs på denna kurs så att andra kurser blir försenade och stör längre in i tiden. Vi har satt in åtgärder mot detta genom att minst en gång i veckan ha möte och diskutera ens egna deadlines för veckan i alla kurser. Det gör det lättare att se vem som kan göra vad när.

\section{Tidsplanering vecka för vecka}
Projektplanen uppdelas i veckor för lättare översikt. För varje vecka presenteras
 en mer specificerad uppskattning av tidsåtgång för aktiviteter som behöver vara
 klara innan deadline. Tidsåtgången för deadlinen inkluderar också tiden för dess
 aktiviteter. För de aktiviteter som avklarats visas också den verkliga tidsåtgången
 samt datum för färdigställande. Varje veckas deadlines och milstolpar presenteras
 i tabeller. Ifall det saknas deadlines för någon vecka så saknas också tabell. Observera att med tidsåtgång så menas den sammanlagda tiden samtliga projektmedlemmar lagt ner tillsammans.

 Då projektgruppens medlemmar har vitt skilda dygnsrytmer är det onödigt att bestämma dag och tid för samtliga aktiviteter i förväg. Det räcker med att veta inom vilken vecka aktiviteter ska vara klara. Tidigare inlämnade dokument samt denna projektplan är bevis tydliga nog för att så är fallet.

 \subsection*{Vecka 37 - Planeringsdokument}
 \addcontentsline{toc}{subsection}{Vecka 37 - Planeringsdokument}
 
\begin{tabularx}{\linewidth}{|l|l|X|l|l|l|l|}
	\hline
	Datum & Typ           & Beskrivning        & Uppskattad tid & Tidsåtgång & Kännedom & Prio \\ [0.5ex]
	\hline                                             
	Tors 10/9  & Hård deadline & Planeringsdokument & 6 tim             & 7 tim         & God      & 1    \\
	\hline
\end{tabularx}

Tidsåtgången för planeringsdokumentet utgjordes av 7 timmar. Av dessa ägnades:
\begin{itemize}
	\item Tisdag 8/9:
	\begin{itemize}
		\item 3 tim sammanställande av vad ett planeringsdokument ska innehålla.
		\item 2 tim skrivande av dokumentet.
	\end{itemize}
	\item Onsdag 9/9:
	\begin{itemize}
		\item 30 min åt att figurera ut hur latex dokumentet skulle kompileras.
		\item 1 tim att ta reda på hur en git \texttt{-{}-}hard-reset reverseras.
		\item 30 min att figurera ut hur tabeller skrivs i latex och få dem rätt positionerade i texten.\\
	\end{itemize}

      \end{itemize}

      Ambitionsmål för veckan:
      \begin{itemize}
      \item Vara klar med Planeringsdokumentet en dag innan deadline. - Blev klara den 9/9.
      \item Förstå hur Flask, Python3 och Jinja2 hänger ihop. - I och med tidigare inlämning lades Torsdagen 10/9 på flask tutorials.
      \end{itemize}
      

\newpage


\subsection*{Vecka 38 - Installationsmanual och LOFI}
\addcontentsline{toc}{subsection}{Vecka 38 - Installationsmanual och LOFI}
\begin{tabularx}{\linewidth}{|l|l|l|l|l|l|l|}
	\hline
	Datum & Typ           & Beskrivning                       & Uppskattad tid & Tidsåtgång    & Kännedom & Prio\\
	\hline                                                    
	Tors 17/9  & Hård deadline &\makecell[tl]{Grundläggande \\Installationsmanual} & 4 tim             & 9 tim 15 min  & God      & 1\\
	\hline                                                    
          & Hård deadline & Lofi-prototyp                     & 6 tim             & 12 tim 29 min & God      & 1\\
	\hline
\end{tabularx}

Den Grundläggande installationsmanualen färdigställdes på 9 tim och 5 min:
\begin{itemize}
	\item Måndag 14/9:
	\begin{itemize}
		\item 1 tim 40 min Dokumentet skrevs.
		\item 20 min Bash-script som tog bort onödiga latex filer skrevs.
		\item 2 tim lära sig kompilera .tex dokument innehållandes bilder samt lägga in dem och läsa på om paket.
	\end{itemize}
	\item Tisdag 15/9:
	\begin{itemize}
		\item 4 tim Dokumentet skrevs.
		\item 1 tim att ta reda på alla kommandon som visar att installtion gått som förväntat.
		\item 15 min Dokumentet rättades och bildernas plats justerades.
	\end{itemize}
\end{itemize}

Lofi-prototypen färdigställdes på 12 tim och 29 min:
\begin{itemize}
	\item Tisdag 15/9:
	\begin{itemize}
		\item 1 tim HTML Skrevs för förstasidan.
		\item 4 tim CSS Skrevs för förstasidan.
	\end{itemize}
	\item Onsdag 16/9:
	\begin{itemize}
		\item 2 tim Utkast snabbskissades för samtliga sidor.
		\item 3 tim Skissarna förbättrades och renskrevs för hand.
	\end{itemize}
	\item Torsdag 17/9:
	\begin{itemize}
		\item 2 tim 9 min De renskrivna skisserna scannades in och sammanställdes med förklaringar i ett latex dokument.
		\item 20 min Dokumentet rättades och bildernas plats justerades.
	\end{itemize}
\end{itemize}


Ambitionsmål för veckan:
\begin{itemize}
  \item Kunna presentera LOFI-prototypen i HTML5 och CSS3 format. - På grund av sjukdom kunde bara förstasidan / färdigställas.
  \end{itemize}
  
\newpage


  
\subsection*{Vecka 39 - Projektplan och Gemensam Installationsmanual}
\addcontentsline{toc}{subsection}{Vecka 39 - Projektplan och Gem. Installationsmanual}
\begin{tabularx}{\linewidth}{|l|l|X|l|l|l|l|}
	\hline
	Datum & Typ           & Beskrivning                        & Uppskattad tid & Tidsåtgång     & Kännedom & Prio \\ [0.5ex]
	\hline                                     
	Tors 24/9  & Hård deadline & \makecell[tl]{1:a Version\\ Gemensam \\Installationsmanual} & 24 tim            & 8 tim    & God      & 1    \\
	\hline                                     
              & Hård deadline & \makecell[tl]{Projektplan: \\Första utkast}         & 24 tim            & 22 tim 30 min & God      & 1    \\
  \hline
 & Milstolpe & \makecell[tl]{Projektplan: klar} & & & &\\
	\hline
\end{tabularx}

Egna insatser 1:a Versionen Gemensam installationsmanual färdigställdes på 8 timmar:
\begin{itemize}
	\item Måndag 21/9:
	\begin{itemize}
		\item 20 min Skapa branch från master för development.
		\item 5 tim 31 min Skriva READMEs i markdown i både hemkatalagen och manuals katalogen. Fixa lättförståelig katalogstruktur. Felsöka initiella svårigheter att push:a upp filer.
	\end{itemize}
	\item Tisdag 22/9:
	\begin{itemize}
                \item 3 tim skriva issues, hjälp med felsökning av andras problem och evaluera färdiga issues.
        \end{itemize}
        \item Onsdag 23/9:
        \begin{itemize}
                \item 50 min Kompletterat inledningen och hjälpt att lösa andras merge conflicts.\\
	\end{itemize}
\end{itemize}

	Projektplan 1:a utkast färdigställdes på 20 timmar 30 minuter:
\begin{itemize}
	\item Torsdag 17/9:
	\begin{itemize}
		\item 1 tim Minst 2-3 aktiviteter definierades inför varje deadline.
	\end{itemize}
	\item Tisdag 22/9:
	\begin{itemize}
		\item 4 tim Aktiviteterna för varje deadline skrevs in. Projektplanens upplägg bestämdes.
	\end{itemize}
	\item Onsdag 23/9:
	\begin{itemize}
		\item 2 tim Anteckningar renskrevs och sorterades in i dokumentet.
		\item 5 tim 30 min Reformaterade tabeller och listor.
		\item 30 min Lade till ytterligare information om tidsåtgång.
                \item 2 tim Riskbedömning och åtgärder skrevs in i projektplanen.
	\end{itemize}
        \item Torsdag 24/9:
        \begin{itemize}
		\item 2 tim Formaterade tabeller igen. Lade till information om prioritering.
        \item 2 tim Ambitionsmål bestämdes och adderades till respektive vecka.
        \item 2 tim Rättade till strukturen och skrev in presentationslagret i V.41 samt datalagret i V.40.
        \item 1 tim Rättade till oegentligheter i tabeller med paketet makecell och skrev tydligare rubriker.
        \item 1 tim Skrev dit milstolpar och såg till att rubriker fortfarande finns med i innehållsförteckningen, men att dess nummer inte skrivs ut i dokumentet.
	\end{itemize}
      \end{itemize}

Ambitionsmål för veckan:
\begin{itemize}
\item Bli helt klar med Gemensam Installationsmanual.
  \item 1 tim Göra projektplanens struktur estetiskt tilltalande.
  \end{itemize}

\newpage

\subsection*{Vecka 40 - Datalagret och Korrigeringar}
\addcontentsline{toc}{subsection}{Vecka 40 - Datalagret och Korrigeringar}
\begin{tabularx}{\linewidth}{|l|l|X|l|l|l|l|}
	\hline
	Datum & Typ           & Beskrivning                                         & Uppskattad tid & Tidsåtgång & Kännedom    & Prio \\ [0.5ex]
	\hline                                                      
	Tors 1/10  & Hård deadline  & \makecell[tl]{Bidra med icke-trivial \\förbättring: gem. \\installationsmanual / \\databas tester} & 1h             &            & God & 2    \\
	\hline                                                      
          & Hård deadline  & \makecell[tl]{Korrigera brister: \\installationsmanualen}  & 0-2 tim           &            & God       & 3    \\
	\hline                                                      
          & Hård deadline  & \makecell[tl]{Korrigera Brister: \\Projektplanen}                   & 0-2 tim           &            & God    & 2    \\
        \hline
          & Milstolpe & \makecell[tl]{Datalagret: Klart} & & & &\\
        \hline
        Fre 2/10 & Hård deadline & Datalagret Godkänt                                  & 24 tim            &            & Vag         & 1    \\
	\hline
\end{tabularx}

Bidra med icke-trivial förbättring - installationsmanual
\begin{itemize}
\item Tisdag 22/9:
  \begin{itemize}
  \item 8 tim Skapa branch av master, skapa issues att göra, skriva instruktioner i READMEs i markdown, komplettera inledning.
  \end{itemize}
  \end{itemize}

Datalagret Godkänt
\begin{itemize}
\item 2 tim Implementera load funktion i datalagret.
\item 1 tim Implementera get\_project\_count funktion i datalagret.
\item 1 tim Implementera  get\_project funktion i datalagret.
\item 3 tim Implementera search funktion i datalagret.
\item 2 tim Implementera get\_techniques funktion i datalagret.
\item 2 tim Implementera get\_technique\_stats funktion i datalagret.
\end{itemize}

Ambitionsmål för veckan:
\begin{itemize}
\item 1 tim Läsa på om och uppskatta tid för att implementera en sqlite3 databas för att uppgradera portfolions back-end senare.
  \end{itemize}

\newpage
  
  \subsection*{Vecka 41 - Presentationslagret}
  \addcontentsline{toc}{subsection}{Vecka 41 - Presentationslagret}
  \begin{tabularx}{\linewidth}{|l|l|X|l|l|l|l|}
	\hline
	Datum        & Typ       & Beskrivning                         & Uppskattad tid & Tidsåtgång & Kännedom & Prio \\ [0.5ex]
	\hline                                             
        Fre 9/10 & Milstolpe & \makecell[tl]{Presentationslagret: \\ Klart}         &            &          & Vag   & 1 \\
	\hline                                             
\end{tabularx}
Presentationslagret
\begin{itemize}
  \item 1 tim Diskutera potentiella ändringar av LOFI:n. 
  \item 4 tim Fixa temasida som agerar template åt alla andra sidor samt lära sig mer om Jinja2, HTML5 och CSS3.
  \item 3 tim Statisk eller dynamisk sida med profilbild, bild för senaste projekt och 'Om mig' text på / sidan.
  \item 5 tim Dynamisk sida som sorterar och listar projekten på /list. Knappar som sorterar sökresultaten. Varje sökresultat har lite beskrivning och liten bild.                  
  \item 3 tim Visar fullständig infosida för specifikt proj med id på /project/id. Samtliga projektsidor.
  \item 3 tim Sammanställning av alla project baserat på använda tekniker på /techniques.
\end{itemize}

Ambitionsmål för veckan:
\begin{itemize}
\item 5 tim Kunna visa på tekniksidan i hur stor omfattning olika tekniker använts för ett viss projekt.
\item 2 tim Ha ett bildspel innehållandes projektbilder på förstasidan.
\end{itemize}

\subsection*{Vecka 42 - Systemdokumentation, -demonstration och Portfolio publicering}
\addcontentsline{toc}{subsection}{Vecka 42 - Systemdokumentation, -demonstration och Portfolio publicering}
\begin{tabularx}{\linewidth}{|l|l|X|l|l|l|l|}
	\hline
	Datum          & Typ           & Beskrivning                  & Uppskattad tid       & Tidsåtgång   & Kännedom & Prio \\ [0.5ex]
	\hline                                             
        Tors 15/10 & Hård deadline & \makecell[tl]{Portfolion    \\ Publicerad}          & -            &          & Vag   & 1 \\
	\hline                                             
                   & Hård deadline & Systemdemonstration          & 3 tim                &              & Vag      & 1    \\
	\hline                                             
                   & Hård deadline & \makecell[tl]{1:a Versionen \\ Systemdokumentation} & 9 tim 30 min &          & Vag   & 3 \\
	\hline
\end{tabularx}

1:a Versionen Systemdokumentation
\begin{itemize}
  \item 1 tim Läsa på om vad systemdokumentationen ska innehålla.
  \item 1 tim Mindmap samt skiss för översiktsbild.
  \item 1 tim Sekvensdiagram
  \item 1 tim Skriva dokumentation kring felhantering.
  \item 1 tim Beskrivning av metoder och program som används vid felsökning.
  \item 2 tim Skriva ett första utkast.
  \item 2 tim Renskriva Systemdokumentation
  \item 30 min Fixa bilders position, rättstavning, 
\end{itemize}

Portfolion Publicerad
\begin{itemize}
\item 1-2 tim Reda ut potentiella buggar.
  \end{itemize}

Ambitionsmål för veckan:
\begin{itemize}
\item 4 tim Gjort portföljerna personliga.
\end{itemize}

\newpage

\subsection*{Vecka 43 - Reflektionsblad och Testdokumentation}
\addcontentsline{toc}{subsection}{Vecka 43 - Reflektionsblad och Testdokumentation}
\begin{tabularx}{\linewidth}{|l|l|X|l|l|l|l|}
	\hline
	Datum      & Typ           & Beskrivning                                        & Uppskattad tid   & Tidsåtgång & Kännedom & Prio \\ [0.5ex]
	\hline                                                                             
	Tors 22/10 & Hård deadline & Testdokumentation inlämnad                         & 3 tim 30 min     &            & Vag      & 1    \\
	\hline                                                                             
               & Hård deadline & \makecell[tl]{Individuellt                        \\ Reflektionsblad} & 5 tim      &          & Vag   & 1 \\
	\hline                                                                             
               & Hård deadline & Korrigerat event. brister i systemdokumentationen  & 0-2 tim          &            & Vag      & 2    \\
	\hline
\end{tabularx}

Testdokumentation inlämnad
\begin{itemize}
  \item 1 tim Generera daterad testlogg där samtliga testfall körts.
  \item 1 tim Dokumentera minst ett relevant testfall per funktionellt krav.
    \item 1 tim min Skriva ihop testdokumentationen.
  \item 30 min Positionera bilder och rätta till text.
\end{itemize}

Individuellt Reflektionsblad
\begin{itemize}
\item 1 tim Plocka ut relevanta delar från dagboken.
  \item 1 tim Skriv ihop delarna i ett latex dokument.
  \item 30 min Positionera potentiella bilder och rätta till text.
  \end{itemize}
  
  Korrigerat eventuella brister i systemdokumentationen.
  \begin{itemize}
  \item 0-2 tim Rätta till potentiella brister.
  \end{itemize}

Ambitionsmål för veckan:
\begin{itemize}
  \item 5 tim Laddat upp portfolio på egen webbserver.
  \end{itemize}

\end{document}

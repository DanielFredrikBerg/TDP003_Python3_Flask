\documentclass{TDP003mall}
\usepackage[utf8]{inputenc}
\usepackage[swedish]{babel}
\usepackage{graphicx}


\newcommand{\version}{Version 1.0}
\author{Daniel Huber, \url{danhu849@liu.se}\\
  Jens Öhnell, \url{jenoh242@liu.se}}
\title{Installationsmanual}
\date{2020-09-14}
\rhead{Daniel Huber\\
Jens Öhnell}



\begin{document}
\projectpage
\section{Revisionshistorik}
\begin{table}[!h]
\begin{tabularx}{\linewidth}{|l|X|l|}
\hline
Ver. & Revisionsbeskrivning & Datum \\\hline
1.0 & Installationsmanual för TDP003 & 200914 \\\hline
\end{tabularx}
\end{table}

\section{Förutsättningar}
Denna manual förutsätter att en Linux distribution är det valda operativsystemet och utgår ifrån Ubuntu 20.04.

\section{Vad behövs?}
För att få att kunna komma åt, använda och underhålla portföljen behövs följande programvaror installerade:
\begin{itemize}
  \item Python3
  \item pip3
  \item emacs
  \item git
  \item flask
  \item jinja2   
\end{itemize}

\section{Installation}
Öppna en terminal med \emph{Ctrl Alt T} och uppdatera och uppgradera systemet till de senaste programvaruuppdateringarna. Skriv \textbf{sudo apt update \&\& sudo apt upgrade} i terminalfönstret och tryck ENTER. Skriv in användarlösenordet vid behov och vänta på att systemet uppdaterar sig.
\begin{figure}[htbp]
  \centerline{\includegraphics[width=\textwidth]{/home/danhu849/Pictures/update_and_upgrade.png}}
  \caption{Visar versionen av Python3 på systemet.}
  \label{fig}
\end{figure}

\subsection{Python3 och pip3}
Python3 ska vara förinstallerat på ubuntu. Kontrollera att det är det genom att skriva \textbf{python3 -V} i terminalen och tryck ENTER. Skriv sedan \textbf{pip3 --version}. Är python3 och pip3 installerade kommer det upp vilket version som finns installerad enligt bild nedan.
\begin{figure}[htbp]
  \centerline{\includegraphics[width=\textwidth]{/home/danhu849/Pictures/python3_and_pip3_check.png}}
  \caption{Visar versionen av Python3 på systemet.}
  \label{fig}
\end{figure}

Ifall python3 eller pip3 \textbf{INTE} är installerat, installera dessa nu genom att skriva \textbf{sudo apt-get install python3.8 python3-pip} i terminalfönstret.
\begin{figure}[htbp]
  \centerline{\includegraphics[width=\textwidth]{/home/danhu849/Pictures/install-Python3.png}}
  \caption{Installera Python3 via terminalen.}
  \label{fig}
\end{figure}
Vänta tills installationen är klar och kolla sedan att Python3 och pip3 blev installerade korrekt genom att skriva in versionskommandona precis som innan.

\subsection{Installera flask och jinja2 med pip3}
pip3 är paketinstalleraren för python3



\section{Felsökning och Underhåll}


\subsection{Tidsåtgång - Presentativ Del}
\begin{tabular}{|c|l|}
  \hline
  5h & Statisk eller dynamisk sida med bilder på /\\
  \hline
  8h & Dynamisk sida som sorterar och listar projekten på /list\\
  \hline
  2h & Visar fullständig infosida för specifikt proj med id på /project/id\\
  \hline
  3h & Sammanställning av alla project baserat på använda tekniker på /techniques\\
  \hline
\end{tabular}

\subsection{Tidsåtgång - Funktioner i datalagret}
Nedan listas den specifika tidsåtgången för implementeringar av respektive funktion i datalagret.

\begin{tabular}{|c|l|}
  \hline
  Tid & Funktion\\
  \hline
  2h & load\\
  \hline
  1h & get\_project\_count\\
  \hline
  1h & get\_project\\
  \hline
  3h & search\\
  \hline
  2h & get\_techniques\\
  \hline
  2h & get\_technique\_stats\\
  \hline
\end{tabular}

\end{document}

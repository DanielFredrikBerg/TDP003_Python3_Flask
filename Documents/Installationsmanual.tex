\documentclass{TDP003mall}
\usepackage[utf8]{inputenc}
\usepackage[swedish]{babel}
\usepackage{graphicx}


\newcommand{\version}{Version 1.0}
\author{Daniel Huber, \url{danhu849@liu.se}\\
  Jens Öhnell, \url{jenoh242@liu.se}}
\title{Installationsmanual}
\date{2020-09-14}
\rhead{Daniel Huber\\
Jens Öhnell}



\begin{document}
\projectpage
\section{Revisionshistorik}
\begin{table}[!h]
\begin{tabularx}{\linewidth}{|l|X|l|}
\hline
Ver. & Revisionsbeskrivning & Datum \\\hline
1.1 & Adderat bilder till styckena & 200915 \\\hline
1.0 & Installationsmanual för TDP003 & 200914 \\\hline
\end{tabularx}
\end{table}

\section{Förutsättningar}
Denna manual förutsätter att en Linux distribution är det valda operativsystemet och utgår ifrån Ubuntu 20.04. Det förutsätts också att användaren förstår att ENTER ska tryckas ned efter varje kommando skrivits in i terminalen. Kommandon är markerade i texten med \textbf{fetstil}.

\section{Vad behövs?}
För att få att kunna komma åt, använda och underhålla portföljen behövs följande programvaror installerade:
\begin{itemize}
  \item Python3
  \item pip3
  \item emacs
  \item git
  \item flask
  \item jinja2   
\end{itemize}

\section{Installation}
Öppna en terminal med \emph{Ctrl Alt T} och uppdatera och uppgradera systemet till de senaste programvaruuppdateringarna. Skriv \textbf{sudo apt update \&\& sudo apt upgrade} i terminalfönstret och tryck ENTER. Skriv in användarlösenordet vid behov och vänta på att systemet uppdaterar sig.

\begin{figure}[h]
  \centerline{\includegraphics[width=8cm]{/home/danhu849/Pictures/update_and_upgrade.png}}
  \caption{Uppdatera och uppgradera alltid innan installation.}
  \label{fig}
\end{figure}

\subsection{Emacs eller annan editor}
Om det redan finns preferenser gällande editor att skriva i installera den på egen hand. Annars är emacs en väldigt bra och open-source editor att använda. Tyvärr får det ej i denna installationsmanual plats att lära sig hur emacs används, men det finns gott om resurser online och en tutorial i själva editorn.

Installera emacs genom att skriva \textbf{sudo apt-get install emacs} i terminalfönstret. Skriv in lösenord vid behov och kolla om emacs är installerad genom att skriva \textbf{emacs \texttt{-{}-}version}
\begin{figure}[h]
  \centerline{\includegraphics[width=8cm]{/home/danhu849/Pictures/emacs_version.png}}
  \caption{Visar versionen av emacs på systemet.}
  \label{fig}
\end{figure}

\subsection{Git}
I samma öppnade terminal kolla om git är installerad genom att skriva in \textbf{git \texttt{-{}-}version}. Om git inte är installerad, installera git nu genom att skriva in \textbf{sudo apt-get install git} vänta tills installationen är klar och kolla versionen av git med \textbf{git \texttt{-{}-}version}.
\begin{figure}[h]
  \centerline{\includegraphics[width=8cm]{/home/danhu849/Pictures/git-V_install.png}}
  \caption{Visar versionen av git på systemet.}
  \label{fig}
\end{figure}

\subsection{Python3 och pip3}
Python3 ska vara förinstallerat på Ubuntu 20.04. Kontrollera att det är det genom att skriva \textbf{python3 -V} i terminalen. Skriv sedan \textbf{pip3 \texttt{-{}-}version}. Är python3 och pip3 installerade kommer det upp vilket version som finns installerad enligt bild nedan.
\begin{figure}[h]
  \centerline{\includegraphics[width=8cm]{/home/danhu849/Pictures/python3_and_pip3_check.png}}
  \caption{Visar versionen av Python3 och pip3 på systemet.}
  \label{fig}
\end{figure}

Ifall python3 eller pip3 \textbf{INTE} är installerat, installera dessa nu genom att skriva \textbf{sudo apt-get install python3.8 python3-pip} i terminalfönstret.

\begin{figure}[h]
  \centerline{\includegraphics[width=8cm]{/home/danhu849/Pictures/install-Python3.png}}
  \caption{Installera Python3 via terminalen.}
  \label{fig}
\end{figure}
Vänta tills installationen är klar och kolla sedan att Python3 och pip3 blev installerade korrekt genom att skriva in versionskommandona precis som innan.

\subsection{Installera flask och jinja2 med pip3 i virtuell miljö}
pip3 är paketinstalleraren för python3 och olika paket adderar olika funktionalitet. Ett problem som dock kan uppstå är när olika paket inte fungerar med varandra. För att undvika att något sådant uppstår kommer det först upprättas en virtuell python miljö där paketen sedan installeras. Fördelen med detta är att den virtuella python miljön är avskild den huvudsakliga pythonmiljön på hårddisken. Paket som installeras virtuellt kommer inte kunna komma i konflikt med de paket vi installerar i vår huvudsakliga installation.

Öppna en terminal om denna inte redan är öppen och gå in i den katalogen där portfolion ska underhållas. Skapa möjligheten att aktivera en virtuell miljö i nuvarande mapp genom att skriva \textbf{python3 -m venv venv} i terminalen. Detta skapar en katalog ''venv'' där alla pip3 relaterade paket kommer installeras. 
\begin{figure}[h]
  \centerline{\includegraphics[width=8cm]{/home/danhu849/Pictures/create_venv_catalog.png}}
  \caption{Skapa en venv katalog i katalogen för portfolion.}
  \label{fig}
\end{figure}
Aktivera sedan den virtuella miljön med \textbf{. venv/bin/activate}. Innan dit användarnamn i terminalen står det nu (venv) vilket betyder att vi nu är i en virtuell miljö och kan installera pythonpaket säkert.
\begin{figure}[h]
  \centerline{}
  \caption{Aktivera den virtuella pythonmiljön.}
  \label{fig}
\end{figure}

Installera först flask med pip3, \textbf{pip3 install flask}. Installera sedan jinja2, \textbf{pip3 install Jinja2}. Kontrollera att de båda installerades genom att skriva först \textbf{pip3 show flask} och se om Location innehåller \path{/den_nuvarande_mappen/venv} och sedan kontrollera samma för jinja2 med  \textbf{pip3 show jinja2} . Vid behov, deaktivera den virtuella miljön genom att skriva deactivate. (venv) som fanns innan framför användarnamnet i terminalfönstret har nu försvunnit.

\begin{figure}[h]
  \centerline{}
  \caption{Visar versionen av Python3 och pip3 på systemet.}
  \label{fig}
\end{figure}

\subsection{Datalagret}

Nu är miljön för att göra och underhålla sin egen portfolio skapad. Det går alldeles utmärkt att push:a upp katalogen som portfolioprojektet ligger i till ett git-repo.



\section{Felsökning och Underhåll}



\end{document}

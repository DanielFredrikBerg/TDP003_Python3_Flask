\documentclass{TDP003mall}
\usepackage[utf8]{inputenc}
\usepackage[swedish]{babel}


\newcommand{\version}{Version 1.0}
\author{Daniel Huber, \url{danhu849@liu.se}\\
  Jens Öhnell, \url{jenoh242@liu.se}}
\title{Testdokumentation}
\date{2020-10-16}
\rhead{Daniel Huber\\
Jens Öhnell}


\begin{document}
\projectpage
\section{Revisionshistorik}
\begin{table}[!h]
\begin{tabularx}{\linewidth}{|l|X|l|}
\hline
Ver. & Revisionsbeskrivning & Datum \\\hline
1.0 & Testdokumentation för Portfolio TDP003 & 161020 \\\hline
\end{tabularx}
\end{table}


\section{Testspecifikation}
Nedan följer en lista med specifika tester för de krav som ställs på portfoliosystemet och återfinns på kurshemsidan i dokumentet \texttt{Systemspecifikation}. Kraven numreras i x.y där x representerar vilket huvudsakligt område testen är designade mot och y är ett löpnummer för antalet tester under x.

\subsection{Tester för presentationslager}
\subsubsection*{Test 1.1 (Krav 1.1)}
\begin{itemize}
\item[]\textbf{Indata} 
\item[]\textbf{Resultat} 
\end{itemize}
\subsubsection*{Test 1.2 (Krav 1.)}
\begin{itemize}
\item[]\textbf{Indata} 
\item[]\textbf{Resultat} 
\end{itemize}
\subsubsection*{Test 1.3 (Krav 1.)}
\begin{itemize}
\item[]\textbf{Indata} 
\item[]\textbf{Resultat} 
\end{itemize} 
\subsubsection*{Test 1.4 (Krav 1.)}
\begin{itemize}
\item[]\textbf{Indata} 
\item[]\textbf{Resultat} 
\end{itemize}
\subsubsection*{Test 1.5 (Krav 1.)}
\begin{itemize}
\item[]\textbf{Indata} 
\item[]\textbf{Resultat} 
\end{itemize}
\subsubsection*{Test 1.6 (Krav 1.)}
\begin{itemize}
\item[]\textbf{Indata} 
\item[]\textbf{Resultat} 
\end{itemize}
\subsubsection*{Test 1.7 (Krav 1.)}
\begin{itemize}
\item[]\textbf{Indata} 
\item[]\textbf{Resultat} 
\end{itemize}

\subsection{Tester för datalager}
\subsubsection*{Test 2.1 (Krav 2.)}
\begin{itemize}
\item[]\textbf{Indata} 
\item[]\textbf{Resultat} 
\end{itemize}
\subsubsection*{Test 2.2 (Krav 2.)}
\begin{itemize}
\item[]\textbf{Indata} 
\item[]\textbf{Resultat} 
\end{itemize}
\subsubsection*{Test 2.3 (Krav 2.)}
\begin{itemize}
\item[]\textbf{Indata} 
\item[]\textbf{Resultat} 
\end{itemize}
\subsubsection*{Test 2.4 (Krav 2.)}
\begin{itemize}
\item[]\textbf{Indata} 
\item[]\textbf{Resultat} 
\end{itemize}
\subsubsection*{Test 2.5 (Krav 2.)}
\begin{itemize}
\item[]\textbf{Indata} 
\item[]\textbf{Resultat} 
\end{itemize}
\subsubsection*{Test 2.6 (Krav 2.)}
\begin{itemize}
\item[]\textbf{Indata} 
\item[]\textbf{Resultat} 
\end{itemize}
\subsubsection*{Test 2.7 (Krav 2.)}
\begin{itemize}
\item[]\textbf{Indata} 
\item[]\textbf{Resultat} 
\end{itemize}
\subsubsection*{Test 2.8 (Krav 2.)}
\begin{itemize}
\item[]\textbf{Indata} 
\item[]\textbf{Resultat} 
\end{itemize}
\subsubsection*{Test 2.9 (Krav 2.)}
\begin{itemize}
\item[]\textbf{Indata} 
\item[]\textbf{Resultat} 
\end{itemize}
\subsubsection*{Test 2.10 (Krav 2.)}
\begin{itemize}
\item[]\textbf{Indata} 
\item[]\textbf{Resultat} 
\end{itemize}

\subsection{Icke-funktionella Krav}
\subsubsection*{Test 3.1 (Krav 3.)}
\begin{itemize}
\item[]\textbf{Indata} 
\item[]\textbf{Resultat} 
\end{itemize}
\subsubsection*{Test 3.2 (Krav 3.)}
\begin{itemize}
\item[]\textbf{Indata} 
\item[]\textbf{Resultat} 
\end{itemize}
\subsubsection*{Test 3.3 (Krav 3.)}
\begin{itemize}
\item[]\textbf{Indata} 
\item[]\textbf{Resultat} 
\end{itemize}
\subsubsection*{Test 3.4 (Krav 3.)}
\begin{itemize}
\item[]\textbf{Indata} 
\item[]\textbf{Resultat} 
\end{itemize}
\subsubsection*{Test 3.5 (Krav 3.)}
\begin{itemize}
\item[]\textbf{Indata} 
\item[]\textbf{Resultat} 
\end{itemize}
\subsubsection*{Test 3.6 (Krav 3.)}
\begin{itemize}
\item[]\textbf{Indata} 
\item[]\textbf{Resultat} 
\end{itemize}
\subsubsection*{Test 3.7 (Krav 3.)}
\begin{itemize}
\item[]\textbf{Indata} 
\item[]\textbf{Resultat} 
\end{itemize}
\subsubsection*{Test 3.8 (Krav 3.)}
\begin{itemize}
\item[]\textbf{Indata} 
\item[]\textbf{Resultat} 
\end{itemize}
\subsubsection*{Test 3.9 (Krav 3.)}
\begin{itemize}
\item[]\textbf{Indata} 
\item[]\textbf{Resultat} 
\end{itemize}
\subsubsection*{Test 3.10 (Krav 3.)}
\begin{itemize}
\item[]\textbf{Indata} 
\item[]\textbf{Resultat} 
\end{itemize}


%Formateringsexempel

\begin{itemize}
    \item 
    \item 
    \item 
    \item 
    \item 
    \item 
\end{itemize}


\begin{figure}[h]
  \centerline{\includegraphics[width=\textwidth, height=10cm]{}}
  \caption{ \label{fig:2}}
\end{figure}

\ref{fig:2} 



\end{document}

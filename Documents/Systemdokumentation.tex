\documentclass{TDP003mall}
\usepackage[utf8]{inputenc}
\usepackage[swedish]{babel}


\newcommand{\version}{Version 1.0}
\author{Daniel Huber, \url{danhu849@liu.se}\\
  Jens Öhnell, \url{jenoh242@liu.se}}
\title{Systemdokumentation}
\date{2020-10-15}
\rhead{Daniel Huber\\
Jens Öhnell}



\begin{document}
\projectpage
\section{Revisionshistorik}
\begin{table}[!h]
\begin{tabularx}{\linewidth}{|l|X|l|}
\hline
Ver. & Revisionsbeskrivning & Datum \\\hline
1.0 & Systemdokumentation för Portfolio TDP003 & 151020 \\\hline
\end{tabularx}
\end{table}


\section{Översiktsbild}
Hemsidans syfte är att presentera information om hemsidans ägare, och presentera project som denna har gjort på ett sökbart sätt. Den består av ett datalager samt ett presentationslager.

\subsection{Datalagret}
Datalagrets syfte är att gör information om olika projekt åtkommlig och sökbar. Det är skrivet i Python 3, och söker mot en json-fil.

\subsection{Presentationslagret}
Presentationslagrets syfte är att göra informationen tillgänglig till slutanvändaren på ett enkelt och intuitivt sätt. Det är skrivet i Python 3 med ramverket Flask samt templatemotorn Jinja2.

\section{Specifikation - Presentationslagret}
Skapandet av hemsidan har utgått från en kravspecifikation. Denna har varit vägledande i vilken funktionalitet sidan skall ha. Den fulla specifikationen kan hittas i dokumentet \textit{Systemspecifikation av portfoliosystem}.

Sammanfattningsvis skall dock följande ingå i hemsidan:
\begin{itemize}
    \item En förstasida med bilder.
    \item En söksida som gör det möjligt att söka och sortera på diverse fält för projekten.
    \item En tekniksida som gör det möjligt att sortera projekt på använda tekniker.
    \item Bilder, både "thumbnails" och fullstora, för all projekt.
    \item Begripliga felmedelanden.
    \item Korrekta statuskoder.
\end{itemize}


\section{Sekvensdiagram - Sökning}



\section{Felhantering och Loggar}
Samtliga portfoliohändelser loggas med hjälp av Pythons logging.config modul. Konfigurationen för loggningen definieras i filen <logging.cfg>. Loggar skrevs till filen <portfolio\_log.log> i kronologisk ordning. Senast log längst ner. Båda filerna finns i root katalogen för portfolio-appen. Loggar sparas med datumstämpel, viktighet, namn, tråd och meddelande. Alla loggar skrivs in i logfilen oavsett viktighet, men kan ändras vid behov genom att ändra <level=DEBUG> under <[logger\_root]> i konfigurationsfilen <logging.cfg>. En nivå som filtrerar bort mer än DEBUG rekommenderas ej då händelser som lett upp till kraschen kan ha filtrerats bort. Logfilen finns tillgänglig även när flask inte körs. Vid ny körning av flask skrivs inte portfolio\_log.log över utan nya loggar skrivs in längst ner i filen.

Terminalkommandot tail -f används vid flask run för att i realtid övervaka de senast tillagda loggarna i portfolio\_log.log. tail används med flaggan -f och vid behov flaggan -n.


Beskrivning av metoder och program för felsökning. Print() och utskrifter i portfoliologgen (flask debug mode). Sidan man kommer till vid fel.

De gemensamma enhetstesterna för datalagret
Standardiserade enhetstester för datalagret

\subsection{Lokal Docker Testning}
''Om ni vill får ni gärna ha med hur man startar en lokal Docker-avbild och testar den'' - Filip Strömbäck


\begin{tabular}{|l|l|l|l|l|}
  \hline
  Vecka & Deadline Datum & Uppskattad & Beskrivning & Kunskap \\ [0.5ex]
  \hline
  37 & Torsdag 10/9 & 6h & Planeringsdokument & God\\
  \hline
  & Lördag 12/9 & 3h & Sätta emacs i python-mode & Vag\\
  \hline
  & Söndag 13/9 & 2h & Inkorporera Magit i Emacs & Vag\\
  \hline
  38 & Tisdag 15/9 & 3h & Bekantskap med Seleniumhq.org & Vag\\
  \hline
  & Torsdag 17/9 & 6h & Lofi-prototyp & God\\
  \hline
  & Torsdag 17/9 & 4h & Grundläggande Installations manual & God\\
  \hline
  39 & Torsdag 24/9 & 24h & Projektplan, Första utkast & Vag\\
  \hline
  & Torsdag 24/9 & 24h & 1:a Versionen gemensam installationsmanual & God\\
  \hline
  & Fredag 25/9 & 2h & Flask o Jinja2 Föreläsning & Vag\\
  \hline
  & Söndag 27/9 & 6h & Ha deploy:at första testhemsidan med flask & Vag\\
  \hline
  & Söndag 27/9 & 12h & Fatta Jinja2 & Vag\\
  \hline
  40 & Torsdag 1/10 & 1h & Bidra med icke-trivial förbättring git eller tester & God o inget\\
  \hline
  & Torsdag 1/10 & 1-2h & Korrigera eventuella brister installationsmanualen & okänt\\
  \hline
  & Torsdag 1/10 & 1-2h & Korrigera Brister, Projektplanen & Beror på\\
  \hline
  & Torsdag 1/10 & 24h & Datalagret Godkänt & Vag\\
  \hline
  42 & Torsdag 15/10 & - & Portfolion Publicerad & Vag\\
  \hline
  & Torsdag 15/10 & 3h & Systemdemonstration & Vag\\
  \hline
  & Torsdag 15/10 & 8h & 1:a Versionen Systemdokumentation & Vag\\
  \hline
  43 & Torsdag 22/10 & 8h & Testdokumentation inlämnad & Vag\\
  \hline
  & Torsdag 22/10 & 6h & Individuellt Reflektionsblad & Vag\\
  \hline
  & Torsdag 22/10 & 2-3h & Korrigerat event. brister i systemdokumentationen & Vag\\
  \hline
  \hline
  Alla & & 20min/dag & Dokumentera Dagbok & God\\
  \hline
\end{tabular}

\subsection{Tidsåtgång - Presentativ Del}
\begin{tabular}{|c|l|}
  \hline
  5h & Statisk eller dynamisk sida med bilder på /\\
  \hline
  8h & Dynamisk sida som sorterar och listar projekten på /list\\
  \hline
  2h & Visar fullständig infosida för specifikt proj med id på /project/id\\
  \hline
  3h & Sammanställning av alla project baserat på använda tekniker på /techniques\\
  \hline
\end{tabular}

\subsection{Tidsåtgång - Funktioner i datalagret}
Nedan listas den specifika tidsåtgången för implementeringar av respektive funktion i datalagret.

\begin{tabular}{|c|l|}
  \hline
  Tid & Funktion\\
  \hline
  2h & load\\
  \hline
  1h & get\_project\_count\\
  \hline
  1h & get\_project\\
  \hline
  3h & search\\
  \hline
  2h & get\_techniques\\
  \hline
  2h & get\_technique\_stats\\
  \hline
\end{tabular}

\end{document}

\documentclass{TDP003mall}
\usepackage[utf8]{inputenc}
\usepackage[swedish]{babel}


\newcommand{\version}{Version 1.0}
\author{Daniel Huber, \url{danhu849@liu.se}\\
  Jens Öhnell, \url{jenoh242@liu.se}}
\title{Planeringsdokument}
\date{2020-10-15}
\rhead{Daniel Huber\\
Jens Öhnell}



\begin{document}
\projectpage
\section{Revisionshistorik}
\begin{table}[!h]
\begin{tabularx}{\linewidth}{|l|X|l|}
\hline
Ver. & Revisionsbeskrivning & Datum \\\hline
1.0 & Systemdokumentation för Portfolio TDP003 & 151020 \\\hline
\end{tabularx}
\end{table}


\section{Översiktsbild}

\section{Specifikation - Presentationslagret}

\section{Sekvensdiagram - Sökning}



\section{Felhantering}
Samtliga portfolio händelser loggas med hjälp av logging.config modulen. Konfigurationen för loggningen definieras i logging.cfg filen. 

Beskrivning av metoder och program för felsökning. Print() och utskrifter i portfoliologgen (flask debug mode). Sidan man kommer till vid fel.

De gemensamma enhetstesterna för datalagret
Standardiserade enhetstester för datalagret



\begin{tabular}{|l|l|l|l|l|}
  \hline
  Vecka & Deadline Datum & Uppskattad & Beskrivning & Kunskap \\ [0.5ex]
  \hline
  37 & Torsdag 10/9 & 6h & Planeringsdokument & God\\
  \hline
  & Lördag 12/9 & 3h & Sätta emacs i python-mode & Vag\\
  \hline
  & Söndag 13/9 & 2h & Inkorporera Magit i Emacs & Vag\\
  \hline
  38 & Tisdag 15/9 & 3h & Bekantskap med Seleniumhq.org & Vag\\
  \hline
  & Torsdag 17/9 & 6h & Lofi-prototyp & God\\
  \hline
  & Torsdag 17/9 & 4h & Grundläggande Installations manual & God\\
  \hline
  39 & Torsdag 24/9 & 24h & Projektplan, Första utkast & Vag\\
  \hline
  & Torsdag 24/9 & 24h & 1:a Versionen gemensam installationsmanual & God\\
  \hline
  & Fredag 25/9 & 2h & Flask o Jinja2 Föreläsning & Vag\\
  \hline
  & Söndag 27/9 & 6h & Ha deploy:at första testhemsidan med flask & Vag\\
  \hline
  & Söndag 27/9 & 12h & Fatta Jinja2 & Vag\\
  \hline
  40 & Torsdag 1/10 & 1h & Bidra med icke-trivial förbättring git eller tester & God o inget\\
  \hline
  & Torsdag 1/10 & 1-2h & Korrigera eventuella brister installationsmanualen & okänt\\
  \hline
  & Torsdag 1/10 & 1-2h & Korrigera Brister, Projektplanen & Beror på\\
  \hline
  & Torsdag 1/10 & 24h & Datalagret Godkänt & Vag\\
  \hline
  42 & Torsdag 15/10 & - & Portfolion Publicerad & Vag\\
  \hline
  & Torsdag 15/10 & 3h & Systemdemonstration & Vag\\
  \hline
  & Torsdag 15/10 & 8h & 1:a Versionen Systemdokumentation & Vag\\
  \hline
  43 & Torsdag 22/10 & 8h & Testdokumentation inlämnad & Vag\\
  \hline
  & Torsdag 22/10 & 6h & Individuellt Reflektionsblad & Vag\\
  \hline
  & Torsdag 22/10 & 2-3h & Korrigerat event. brister i systemdokumentationen & Vag\\
  \hline
  \hline
  Alla & & 20min/dag & Dokumentera Dagbok & God\\
  \hline
\end{tabular}

\subsection{Tidsåtgång - Presentativ Del}
\begin{tabular}{|c|l|}
  \hline
  5h & Statisk eller dynamisk sida med bilder på /\\
  \hline
  8h & Dynamisk sida som sorterar och listar projekten på /list\\
  \hline
  2h & Visar fullständig infosida för specifikt proj med id på /project/id\\
  \hline
  3h & Sammanställning av alla project baserat på använda tekniker på /techniques\\
  \hline
\end{tabular}

\subsection{Tidsåtgång - Funktioner i datalagret}
Nedan listas den specifika tidsåtgången för implementeringar av respektive funktion i datalagret.

\begin{tabular}{|c|l|}
  \hline
  Tid & Funktion\\
  \hline
  2h & load\\
  \hline
  1h & get\_project\_count\\
  \hline
  1h & get\_project\\
  \hline
  3h & search\\
  \hline
  2h & get\_techniques\\
  \hline
  2h & get\_technique\_stats\\
  \hline
\end{tabular}

\end{document}
